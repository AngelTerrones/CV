\documentclass[10pt]{letter}
\usepackage{fullpage}                           % Gives 1-in margins
\usepackage[left=2.5cm,top=2.5cm,right=2.5cm,bottom=2.5cm]{geometry}
\usepackage{mathtools,amssymb,amsfonts}
\usepackage[utf8]{inputenc}
\usepackage[spanish]{babel}

%\usepackage[shortlabels]{enumitem}
\usepackage{paralist}
\usepackage{varioref}
\usepackage[%
        colorlinks=true,urlcolor=blue,linkcolor=blue,
        pdfpagelabels=true,hypertexnames=true,
        plainpages=false,naturalnames=false,
        ]{hyperref}
\labelformat{enumi}{[#1]}

\newcommand\doilink[1]{\href{http://dx.doi.org/#1}{#1}}
\newcommand\doi[1]{doi:\doilink{#1}}
\newcommand\email[1]{\href{mailto:#1}{\texttt{#1}}}

\signature{\vspace{-0.35in}Ángel Jesús Terrones Bermúdez\\}
\address{MSc.~Ángel Jesús Terrones Bermúdez\\
        Calle Teodosio Angellino, \\
        Residencias Barcelona,\\
        Torre A, piso 11, apto. 113. \\
        Cúa, Edo. Miranda. Venezuela. \\
         \hfill {\texttt{+58-239-2126980}/\email{angelterrones@gmail.com}}}

\begin{document}
\begin{letter}{Comisión Académica del programa de Doctorado\\
    Departamento de Informática e Ingeniería de Sistemas\\
    Universidad de Zaragoza}

\opening{Estimados miembros de la Comisión Académica:}

Me dirijo a ustedes con el fin de aplicar al programa de Doctorado en
Ingeniería de Sistemas e Informática de la Universidad de Zaragoza.

Me gradué del programa de Maestría en Ingeniería Electrónica opción
Mecatrónica en la Universidad Simón Bolívar, Venezuela, en el año 2013.
Actualmente soy profesor en el área de circuitos digitales en la misma
Universidad, donde además de obtener experiencia en enseñanza, supervicé
diversos trabajos de grado en el área de robótica móvil.

El problema de navegación de un sistema compuesto por múltiples robots es un
tema fascinante para mi debido a los retos que implica, y esto me motivó a
proponerlo y desarrollarlo como mi trabajo de grado para la obtención del titulo
de maestría. En dicho trabajo desarrollé un conjunto de reglas simples, y de
naturaleza distribuida, que permiten a un grupo de vehículos terrestres
desplazarse en un ambiente desconocido y en forma conjunta,
evitando colisiones entre si y contra obstáculos presentes en el ambiente.

En Abril de este año tuve conocimiento del proyecto titulado
``Coordinación y Visión Distribuida de Sistemas Multi-Robot para Exploración
Remota'', el cual es desarrollado por el grupo de Robótica, Percepción y Tiempo
Real (RoPeRT) de la Universidad de Zaragoza y es coordinado por los profesores
Carlos Sagüés y Eduardo Montijano. El poder participar en este proyecto como
estudiante de doctorado me ofrece una excelente oportunidad para trabajar en el
área que me atrae, y así poder desarrollar mi experticia en el área de sistemas
multi-robot.

Gracias por su tiempo y consideración.

\closing{Atentamente,}


\end{letter}
\end{document}
