\documentclass[10pt]{letter}
\usepackage{fullpage}                           % Gives 1-in margins
\usepackage[left=2.5cm,top=2.5cm,right=2.5cm,bottom=2.5cm]{geometry}
\usepackage{mathtools,amssymb,amsfonts}
\usepackage[utf8]{inputenc}
\usepackage[spanish]{babel}

%\usepackage[shortlabels]{enumitem}
\usepackage{paralist}
\usepackage{varioref}
\usepackage[%
        colorlinks=true,urlcolor=blue,linkcolor=blue,
        pdfpagelabels=true,hypertexnames=true,
        plainpages=false,naturalnames=false,
        ]{hyperref}
\labelformat{enumi}{[#1]}

\newcommand\doilink[1]{\href{http://dx.doi.org/#1}{#1}}
\newcommand\doi[1]{doi:\doilink{#1}}
\newcommand\email[1]{\href{mailto:#1}{\texttt{#1}}}

\signature{\vspace{-0.35in}Ángel Jesús Terrones Bermúdez\\}
\address{MSc.~Ángel Jesús Terrones Bermúdez\\
        Calle Teodosio Angellino, \\
        Residencias Barcelona,\\
        Torre A, piso 11, apto. 113. \\
        Cúa, Edo. Miranda. Venezuela. \\
         \hfill {\texttt{+58-239-2126980}/\email{angelterrones@gmail.com}}}

\begin{document}
\begin{letter}{Comisión Académica del programa de Doctorado\\
    Departamento de Informática e Ingeniería de Sistemas\\
    Universidad de Zaragoza}

\opening{Estimados miembros de la Comisión Académica:}

En forma respetuosa, me dirijo a ustedes con el fin de solicitar la admisión al
programa de Doctorado en Ingeniería de Sistemas e Informática de la Universidad
de Zaragoza.

Me gradué del programa de Maestría en Ingeniería Electrónica opción Mecatrónica
en la Universidad Simón Bolívar, Venezuela, en el año 2013. En la actualidad soy
profesor en el área de circuitos digitales en la misma Universidad, donde además
de obtener experiencia en enseñanza, supervisé diversos trabajos de grado en el
área de robótica, y he realizado trabajos de consultoría en el área de
automatización y robótica.

El problema de navegación de un sistema compuesto por múltiples robots es un
tema fascinante para mi debido a los retos que implica, así como sus posibles
aplicaciones en tareas de exploración, búsqueda, inspección y asistencia en
tareas de minería o agricultura, entre otros ejemplos.
Estas razones me motivaron a elegir como línea de investigación para mi trabajo
de grado en la maestría el área de control de sistemas multi-robot. Para mi
trabajo de grado, desarrollé un conjunto de reglas simples y de
naturaleza distribuida que permiten a un grupo de vehículos terrestres
desplazarse en un ambiente desconocido y en forma conjunta, evitando colisiones
entre si y contra obstáculos presentes en el ambiente.

En Abril de este año obtuve conocimiento del proyecto titulado ``Coordinación y
Visión  Distribuida de Sistemas Multi-Robot para Exploración Remota'', el cual es
desarrollado por el grupo de Robótica, Percepción y Tiempo Real (RoPeRT) de la
Universidad  de Zaragoza bajo la tutoría de los profesores Eduardo Montijano y
Rosario Aragüés. Poder participar en este proyecto como estudiante de doctorado
me ofrece una oportunidad excelente para trabajar en el área que me atrae, y así
poder desarrollar mi experticia en el área de sistemas multi-robot.

Gracias por su tiempo y consideración.

\closing{Atentamente,}


\end{letter}
\end{document}
