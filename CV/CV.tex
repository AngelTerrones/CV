%%%%%%%%%%%%%%%%%%%%%%%%%%%%%%%%%%%%%%%%%%%%%%%%%%%%%%%%%%%%%%%%%%%%%%%%
%%%%%%%%%%%%%%%%%%%%%% Simple LaTeX CV Template %%%%%%%%%%%%%%%%%%%%%%%%
%%%%%%%%%%%%%%%%%%%%%%%%%%%%%%%%%%%%%%%%%%%%%%%%%%%%%%%%%%%%%%%%%%%%%%%%

%%%%%%%%%%%%%%%%%%%%%%%%%%%%%%%%%%%%%%%%%%%%%%%%%%%%%%%%%%%%%%%%%%%%%%%%
%% NOTE: If you find that it says                                     %%
%%                                                                    %%
%%                           1 of ??                                  %%
%%                                                                    %%
%% at the bottom of your first page, this means that the AUX file     %%
%% was not available when you ran LaTeX on this source. Simply RERUN  %%
%% LaTeX to get the ``??'' replaced with the number of the last page  %%
%% of the document. The AUX file will be generated on the first run   %%
%% of LaTeX and used on the second run to fill in all of the          %%
%% references.                                                        %%
%%%%%%%%%%%%%%%%%%%%%%%%%%%%%%%%%%%%%%%%%%%%%%%%%%%%%%%%%%%%%%%%%%%%%%%%

%%%%%%%%%%%%%%%%%%%%%%%%%%%% Document Setup %%%%%%%%%%%%%%%%%%%%%%%%%%%%

% Don't like 10pt? Try 11pt or 12pt
\documentclass[10pt]{article}
\RequirePackage[T1]{fontenc}

% LaTeX will typeset using Computer Modern Roman, which a lot of
% non-mathematicians and non-engineers won't like. Also, a few PDF
% viewers may not render CMR very well. Instead, Times New Roman can
% be used. That's what this package does.
\usepackage{times}

% The automated optical recognition software used to digitize resume
% information works best with fonts that do not have serifs. This
% command uses a sans serif font throughout. Uncomment both lines (or at
% least the second) to restore a Roman font (i.e., a font with serifs).
% (NOTE: This requires the times package above)
%\renewcommand{\familydefault}{\sfdefault}

% This is a helpful package that puts math inside length specifications
\usepackage{calc}

% This package helps LaTeX auto-hyphenate hyphenated words if you use
% special hyphens. For example, bio\-/mimicry will properly hyphenate
% ``mimicry'' if necessary.
\usepackage[shortcuts]{extdash}

% Layout: Puts the section titles on left side of page
\reversemarginpar

%
%         PAPER SIZE, PAGE NUMBER, AND DOCUMENT LAYOUT NOTES:
%
% The next \usepackage line changes the layout for CV style section
% headings as marginal notes. It also sets up the paper size as either
% letter or A4. By default, letter was used. If A4 paper is desired,
% comment out the letterpaper lines and uncomment the a4paper lines.
%
% As you can see, the margin widths and section title widths can be
% easily adjusted.
%
% ALSO: Notice that the includefoot option can be commented OUT in order
% to put the PAGE NUMBER *IN* the bottom margin. This will make the
% effective text area larger.
%
% IF YOU WISH TO REMOVE THE ``of LASTPAGE'' next to each page number,
% see the note about the +LP and -LP lines below. Comment out the +LP
% and uncomment the -LP.
%
% IF YOU WISH TO REMOVE PAGE NUMBERS, be sure that the includefoot line
% is uncommented and ALSO uncomment the \pagestyle{empty} a few lines
% below.
%

%% Use these lines for letter-sized paper
\usepackage[paper=letterpaper,
            %includefoot, % Uncomment to put page number above margin
            marginparwidth=1.2in,     % Length of section titles
            marginparsep=.05in,       % Space between titles and text
            margin=1in,               % 1 inch margins
            includemp]{geometry}

%% Use these lines for A4-sized paper
%\usepackage[paper=a4paper,
%            %includefoot, % Uncomment to put page number above margin
%            marginparwidth=30.5mm,    % Length of section titles
%            marginparsep=1.5mm,       % Space between titles and text
%            margin=25mm,              % 25mm margins
%            includemp]{geometry}

%% More layout: Get rid of indenting throughout entire document
\setlength{\parindent}{0in}

% Provides special list environments and macros to create new ones
\usepackage[shortlabels]{enumitem}

% Simpler bibsections for CV sections
% (thanks to natbib for inspiration)
%
% * For lists of references with hanging indents and no numbers:
%
%   \begin{bibsection}
%       \item ...
%   \end{bibsection}
%
% * For numbered lists of references (with hanging indents):
%
%   \begin{bibenum}
%       \item ...
%   \end{bibenum}
%
%   Note that bibenum numbers continuously throughout. To reset the
%   counter, use
%
%   \restartlist{bibenum}
%
%   at the place where you want the numbering to reset.

\makeatletter
\newlength{\bibhang}
\setlength{\bibhang}{1em}
\newlength{\bibsep}
 {\@listi \global\bibsep\itemsep \global\advance\bibsep by\parsep}
\newlist{bibsection}{itemize}{3}
\setlist[bibsection]{label=,leftmargin=\bibhang,%
        itemindent=-\bibhang,
        itemsep=\bibsep,parsep=\z@,partopsep=0pt,
        topsep=0pt}
\newlist{bibenum}{enumerate}{3}
\setlist[bibenum]{label=[\arabic*],resume,leftmargin={\bibhang+\widthof{[999]}},%
        itemindent=-\bibhang,
        itemsep=\bibsep,parsep=\z@,partopsep=0pt,
        topsep=0pt}
\let\oldendbibenum\endbibenum
\def\endbibenum{\oldendbibenum\vspace{-.6\baselineskip}}
\let\oldendbibsection\endbibsection
\def\endbibsection{\oldendbibsection\vspace{-.6\baselineskip}}
\makeatother

%% Reference the last page in the page number
%
% NOTE: comment the +LP line and uncomment the -LP line to have page
%       numbers without the ``of ##'' last page reference)
%
% NOTE: uncomment the \pagestyle{empty} line to get rid of all page
%       numbers (make sure includefoot is commented out above)
%
\usepackage{fancyhdr,lastpage}
\pagestyle{fancy}
%\pagestyle{empty}      % Uncomment this to get rid of page numbers
\fancyhf{}\renewcommand{\headrulewidth}{0pt}
\fancyfootoffset{\marginparsep+\marginparwidth}
\newlength{\footpageshift}
\setlength{\footpageshift}
          {0.5\textwidth+0.5\marginparsep+0.5\marginparwidth-2in}
\lfoot{\hspace{\footpageshift}%
       \parbox{4in}{\, \hfill %
                    \arabic{page} of \protect\pageref*{LastPage} % +LP
%                    \arabic{page}                               % -LP
                    \hfill \,}}

% Finally, give us PDF bookmarks
\usepackage{color,hyperref}
\definecolor{darkblue}{rgb}{0.0,0.0,0.3}
\hypersetup{colorlinks,breaklinks,
            linkcolor=darkblue,urlcolor=darkblue,
            anchorcolor=darkblue,citecolor=darkblue}

\usepackage[utf8]{inputenc}
\usepackage[T1]{fontenc}
\usepackage{lmodern} % load a font with all the characters

%%%%%%%%%%%%%%%%%%%%%%%% End Document Setup %%%%%%%%%%%%%%%%%%%%%%%%%%%%


%%%%%%%%%%%%%%%%%%%%%%%%%%% Helper Commands %%%%%%%%%%%%%%%%%%%%%%%%%%%%

%%% HEADING AT TOP OF CURRICULUM VITAE

% The title (name) with a horizontal rule under it
% (optional argument typesets an object right-justified across from name
%  as well)
%
% Usage: \makeheading{name}
%        OR
%        \makeheading[right_object]{name}
%
% Place at top of document. It should be the first thing.
% If ``right_object'' is provided in the square-braced optional
% argument, it will be right justified on the same line as ``name'' at
% the top of the CV. For example:
%
%       \makeheading[\emph{Curriculum vitae}]{Your Name}
%
% will put an emphasized ``Curriculum vitae'' at the top of the document
% as a title. Likewise, a picture could be included:
%
%   \makeheading[{\includegraphics[height=1.5in]{my_picture}}]{Your Name}
%
% the picture will be flush right across from the name. For this example
% to work, make sure the extra set of curly braces is included. Also
% makes ure that \usepackage{graphicx} is somewhere in the preamble.
\newcommand{\makeheading}[2][]%
        {\hspace*{-\marginparsep minus \marginparwidth}%
         \begin{minipage}[t]{\textwidth+\marginparwidth+\marginparsep}%
             {\large \bfseries #2 \hfill #1}\\[-0.15\baselineskip]%
                 \rule{\columnwidth}{1pt}%
         \end{minipage}}

%%% SECTION HEADINGS

% The section headings. Flush left in small caps down pseudo-margin.
%
% Usage: \section{section name}
\renewcommand{\section}[1]{\pagebreak[3]%
    \vspace{1.3\baselineskip}%
    \phantomsection\addcontentsline{toc}{section}{#1}%
    \noindent\llap{\scshape\smash{\parbox[t]{\marginparwidth}{\hyphenpenalty=10000\raggedright #1}}}%
    \vspace{-\baselineskip}\par}

%%% LISTS

% This macro alters a list by removing some of the space that follows the list
% (is used by lists below)
\newcommand*\fixendlist[1]{%
    \expandafter\let\csname preFixEndListend#1\expandafter\endcsname\csname end#1\endcsname
    \expandafter\def\csname end#1\endcsname{\csname preFixEndListend#1\endcsname\vspace{-0.6\baselineskip}}}

% These macros help ensure that items in outer-type lists do not get
% separated from the next line by a page break
% (they are used by lists below)
\let\originalItem\item
\newcommand*\fixouterlist[1]{%
    \expandafter\let\csname preFixOuterList#1\expandafter\endcsname\csname #1\endcsname
    \expandafter\def\csname #1\endcsname{\let\oldItem\item\def\item{\pagebreak[2]\oldItem}\csname preFixOuterList#1\endcsname}
    \expandafter\let\csname preFixOuterListend#1\expandafter\endcsname\csname end#1\endcsname
    \expandafter\def\csname end#1\endcsname{\let\item\oldItem\csname preFixOuterListend#1\endcsname}}
\newcommand*\fixinnerlist[1]{%
    \expandafter\let\csname preFixInnerList#1\expandafter\endcsname\csname #1\endcsname
    \expandafter\def\csname #1\endcsname{\let\oldItem\item\let\item\originalItem\csname preFixInnerList#1\endcsname}
    \expandafter\let\csname preFixInnerListend#1\expandafter\endcsname\csname end#1\endcsname
    \expandafter\def\csname end#1\endcsname{\csname preFixInnerListend#1\endcsname\let\item\oldItem}}

% An itemize-style list with lots of space between items
%
% Usage:
%   \begin{outerlist}
%       \item ...    % (or \item[] for no bullet)
%   \end{outerlist}
\newlist{outerlist}{itemize}{3}
    \setlist[outerlist]{label=\enskip\textbullet,leftmargin=*}
    \fixendlist{outerlist}
    \fixouterlist{outerlist}

% An environment IDENTICAL to outerlist that has better pre-list spacing
% when used as the first thing in a \section
%
% Usage:
%   \begin{lonelist}
%       \item ...    % (or \item[] for no bullet)
%   \end{lonelist}
\newlist{lonelist}{itemize}{3}
    \setlist[lonelist]{label=\enskip\textbullet,leftmargin=*,partopsep=0pt,topsep=0pt}
    \fixendlist{lonelist}
    \fixouterlist{lonelist}

% An itemize-style list with little space between items
%
% Usage:
%   \begin{innerlist}
%       \item ...    % (or \item[] for no bullet)
%   \end{innerlist}
\newlist{innerlist}{itemize}{3}
    \setlist[innerlist]{label=\enskip\textbullet,leftmargin=*,parsep=0pt,itemsep=0pt,topsep=0pt,partopsep=0pt}
    \fixinnerlist{innerlist}

% An environment IDENTICAL to innerlist that has better pre-list spacing
% when used as the first thing in a \section
%
% Usage:
%   \begin{loneinnerlist}
%       \item ...    % (or \item[] for no bullet)
%   \end{loneinnerlist}
\newlist{loneinnerlist}{itemize}{3}
    \setlist[loneinnerlist]{label=\enskip\textbullet,leftmargin=*,parsep=0pt,itemsep=0pt,topsep=0pt,partopsep=0pt}
    \fixendlist{loneinnerlist}
    \fixinnerlist{loneinnerlist}

%%% EXTRA SPACE

% To add some paragraph space between lines.
% This also tells LaTeX to preferably break a page on one of these gaps
% if there is a needed pagebreak nearby.
\newcommand{\blankline}{\quad\pagebreak[3]}
\newcommand{\halfblankline}{\quad\vspace{-0.5\baselineskip}\pagebreak[3]}

%%% FORMATTING MACROS

% Provides a linked \doi{#1} that links doi:#1 to http://dx.doi.org/#1
\usepackage{doi}
% To change the text before the DOI, adjust this command
%\renewcommand\doitext{doi:}

% Provides a linked \url{#1} that doesn't require escape characters
\usepackage{url}

% You can adjust the style \url{} uses here:
% (options are: same, rm, sf, tt; defaults to tt)
\urlstyle{same}

% For \email{ADDRESS}, links ADDRESS to the url mailto:ADDRESS
% (uncomment to typeset the e\-/mail address in typewriter font;
%  otherwise, will be typeset in the \urlstyle above)
%\DeclareUrlCommand\emaillink{\urlstyle{tt}}
\providecommand*\emaillink[1]{\nolinkurl{#1}}
\providecommand*\email[1]{\href{mailto:#1}{\emaillink{#1}}}

\providecommand\BibTeX{{B\kern-.05em{\sc i\kern-.025em b}\kern-.08em \TeX}}
\providecommand\Matlab{\textsc{Matlab}}

% Custom hyphenation rules for words that LaTeX has trouble with
\hyphenation{bio-mim-ic-ry bio-in-spi-ra-tion re-us-a-ble pro-vid-er Media-Wiki}

%%%%%%%%%%%%%%%%%%%%%%%% End Helper Commands %%%%%%%%%%%%%%%%%%%%%%%%%%%

%%%%%%%%%%%%%%%%%%%%%%%%% Begin CV Document %%%%%%%%%%%%%%%%%%%%%%%%%%%%

\begin{document}
\makeheading{Ángel Jesús Terrones Bermúdez}

\section{Contact Information}

% NOTE: Mind where the & separators and \\ breaks are in the following
%       table. Table is one row made up of three parboxes. The left
%       parbox has address info, the middle parbox has a vertical bar,
%       and the right parbox has phone and electronic contact
%       information.
%
% MACROS: \rcollength is the width of the right column of the table
%             (adjust it to your liking; default is 1.85in).
%         \spacewidth is width of area between left and right boxes.
%
\newlength{\rcollength}\setlength{\rcollength}{1.85in}%
\newlength{\spacewidth}\setlength{\spacewidth}{20pt}
%
\begin{tabular}[t]{@{}p{\textwidth-\rcollength-\spacewidth}@{}p{\spacewidth}@{}p{\rcollength}}%

% Address box
\parbox{\textwidth-\rcollength-\spacewidth}{%
\textit{Address}:\\
Calle Teodosio Angellino, Residencias Barcelona,\\
Torre A, piso 11, apto. 113. \\
Cúa, Edo. Miranda. Venezuela
}

&
% Uncomment to add a vertical bar in middle of contact information
%{\vrule width 0.5pt}
\parbox[m][5\baselineskip]{\spacewidth}{} &

% Non-snail-mail contact information
\parbox{\rcollength}{%
\textit{Mobile:} +58-412-2001428 \\
\textit{E-mail:} \email{angelterrones@gmail.com}\\
\textit{Marital Status}: Single\\
\textit{Age}: 38
}

\end{tabular}

%%
%% In modern CV's, it seems like ``Objective'' is frowned upon. Instead,
%% incorporate it into a well-constructed cover letter. The ``More
%% information'' can go at the end of the CV, but it should not distract
%% from the section giving references available to contact.
%%
%
% \section{Objective}
%
% Placement in an academic position (i.e., faculty, postdoctoral, or
% research scientist) that allows for advanced research in distributed
% complex adaptive systems (i.e., modeling, analysis, design, and
% verification) with a particular focus on the control of engineered
% agents (e.g., for communications, control, software, electronics, and
% sustainability) and the analysis of biological phenomena (e.g.,
% self-organization, ecological rationality)
% \begin{innerlist}
% \item More information and auxiliary documents can be found at\\\url{http://www.tedpavlic.com/facjobsearch/}
% \end{innerlist}

% \section{Research Interests}

% \textbf{Swarm robotics}.

% \textbf{Machine learning}.

\section{Education}

\href{http://www.usb.ve/}{\textbf{Simón Bolívar University}},
Caracas, Venezuela
\begin{outerlist}

\item[] M.S.,
        \href{http://www.usb.ve/}
             {Electronic Engineering. Mechatronics specialization},
             January 2013
        \begin{innerlist}
        \item GPA: 4.92/5
        \item Thesis Topic: \emph{Navigation control for a multi-robot system using distributed artificial intelligence.}
        \item Advisers:
              Gerardo Fernández-Lopez, Ph.D. and
              Leonardo Fermín, MSc.
        \item Area of Study: Swarm navigation and Mobile Robotics
        \end{innerlist}

\item[] B.S.,
        \href{http://www.usb.ve/}
             {Electronic Engineering}, January 2009
        \begin{innerlist}
        \item GPA: 4.26/5
        \item Five year pensum with thesis
        \item Thesis Topic: \emph{Instrumentation of a flexible manipulator}
        \item Adviser: Professor Cecilia Murrugarra.
        \end{innerlist}

\end{outerlist}

\section{Professional Experience}

\textbf{Sento Electronics}
\begin{outerlist}
    \item[] \textit{Software Engineer/Developer}
            \hfill \textbf{January 2024 to April 2024}
            \begin{innerlist}
                \item Development of IoT solutions.
            \end{innerlist}
\end{outerlist}

\textbf{Freelancer developer}
\begin{outerlist}
    \item[] \textit{Software Engineer/Developer}
            \hfill \textbf{Febrero 2022 to Septiembre 2023}
            \begin{innerlist}
                \item Development of different software solutions in the field of machine learning.
            \end{innerlist}
\end{outerlist}

\href{https://hypedev.design/}{\textbf{HyperDev}}
\begin{outerlist}
    \item[] \textit{Software Engineer/Developer}
            \hfill \textbf{January 2021 to December 2022}
            \begin{innerlist}
                \item Development of different IT and engineering solutions.
            \end{innerlist}

\end{outerlist}

\textbf{AlterInfo}, Caracas, Venezuela
\begin{outerlist}
    \item[] \textit{Software Engineer/Developer}%
            \hfill \textbf{August 2018 to December 2020}
            \begin{innerlist}
                \item Development of different IoT solutions.
                \item Development of a tunning system for a PID controller for a synchronous machine.
            \end{innerlist}
\end{outerlist}

\href{http://www.usb.ve/}{\textbf{Simón Bolívar University}}, Caracas, Venezuela
\begin{outerlist}
    \item[] \textit{Assistant Professor}%
            \hfill \textbf{April 2013 to present}
            \begin{innerlist}
                \item Department of Electronics and Circuits. Digital Section.
                \item Instructor in the area of Computer Architecture and Digital Circuits.
                \item Investigation in the area of navigation of multiple mobile
                robots,computer architecture and machine learning.
                \item Supervision of graduate and undergraduate students in engineering.
            \end{innerlist}

    \item[] \textit{Instructor Professor}%
            \hfill \textbf{September 2012 to April 2013}
            \begin{innerlist}
                \item Department of Electronics and Circuits. Digital Section.
                \item Instructor in the area of Computer Architecture and Digital Circuits.
            \end{innerlist}

    \item[] \textit{Academic Assistant}%
            \hfill \textbf{January 2009 to July 2012}
            \begin{innerlist}
                \item Department of Electronics and Circuits. Digital Section.
                \item Asistant to the professors in the area of computer architecture
                and digital circuits.
                \item Design of laboratory exams and activities.
                \item Development of tutorials.
                \item Graded the weekly assignments.
            \end{innerlist}
\end{outerlist}

% \section{Teaching Experience}
%
% \href{http://www.usb.ve/}{\textbf{Simón Bolívar University}}, Caracas, Venezuela
% \begin{outerlist}
%
%     \item[] \textit{Assistant Professor}
%         \hfill \textbf{April 2013 to present}
%         \begin{innerlist}
%             \item Instructor for EC-1723: Digital Circuits.
%             \item Instructor for EC-3084: Microcontrollers Laboratory.
%             \item Instructor for EC-3731: Computer Architecture II.
%             \item Instructor for EC-5723: Genetic Algorithms.
%         \end{innerlist}
%
%     \item[] \textit{Instructor Professor}
%         \hfill \textbf{September 2012 to April 2013}
%         \begin{innerlist}
%             \item Instructor for EC-3192: Electronic Circuits Laboratory.
%             \item Instructor for EC-3881: Projects Laboratory I.
%             \item Instructor for EC-5723: Genetic Algorithms.
%         \end{innerlist}
%     \item[] \textit{Teaching Assistant}
%         \hfill \textbf{January 2009 to July 2012}
%         \begin{innerlist}
%             \item Assistant for EC-2721: Computer Architecture I.
%             \item Assistant for EC-3514: Robotics.
%             \item Assistant for EC-3731: Computer Architecture II.
%             \item Assistant for EC-3881: Projects Laboratory I.
%             \item Assistant for EC-3882: Projects Laboratory II.
%             \item Assistant for EC-3883: Projects Laboratory III.
%         \end{innerlist}
% \end{outerlist}
%
% \section{Professional Memberships}
% R\&D Mechatronics Group (Simón Bolívar University), Member, 2009-present
% %
% \begin{innerlist}
% \item Professor Researcher (2013-present)
% \item Master Thesis Researcher (2009-2012)
% \end{innerlist}

% \section{Publications}

% \begin{bibenum}
%     \item Murrugarra, C., De Castro, O., Terrones, A.
%         A Test Bed to Measre Transverse Deflection of a Flexible Link Manipulator.
%         In: \emph{Applied Computer Science in Engineering. 5th Workshop on
%           Engineering Applications, WEA 2018.},
%         Medellín, Colombia, 17 – 19 October 2018.
%         \doi{10.1007/978-3-030-00353-1_35}

%     \item Ruiz, E., Acuña, R., Certad-H, N., Terrones, Cabrera, M.E.
%         Development of a Control Platform for the Mobile Robot Roomba Using ROS and a Kinect Sensor.
%         In: \emph{Robotics Symposium and Competition (LARS/LARC), 2013 Latin American},
%         Arequipa, Perú, 21 – 27 October 2013.
%         \doi{10.1109/LARS.2013.57}

%     \item Certad-H, N., J., Acuña, R., Terrones, A., Ralev, D., C., Cappelletto and Gireco, J.
%         Study and Improvements in Landmarks Extraction in 2D Range Images Based on an Adaptive
%         Curvature Estimation.
%         In: \emph{Andean Region International Conference (ANDESCON), 2012 VI},
%         Cuenca, Ecuador, 7 – 9 November 2012.\\
%         \doi{10.1109/Andescon.2012.31}

%     \item Mastalli, C., Cappelletto, J., Acuña, R., Terrones, A., and Fernández-López, G.
%         An Imitation Learning Approach For Truck Loading Operations in Backhoe Machines.
%         In: \emph{Adaptive Mobile Robotics: Proceedings of the 15th International
%         Conference on Climbing and Walking Robots and the Support Technologies
%         for Mobile Machines},
%         Baltimore, USA, 23 – 26 July 2012.
%         \doi{10.1142/9789814415958_0104}

%     \item Acuña, R., Terrones, A., Certad-H, N., Fermín-León, L., and Fernández-López, G.
%         Dynamic Potential Field Generation Using Movement Prediction
%         In: \emph{Adaptive Mobile Robotics: Proceedings of the 15th International
%         Conference on Climbing and Walking Robots and the Support Technologies
%         for Mobile Machines},
%         Baltimore, USA, 23 – 26 July 2012.
%         \doi{10.1142/9789814415958_0101}

%     \item Terrones, A., Acuña, R., Certad-H, N., Fermín-León, L., and Fernández-López, G.
%         Local Distributed Control For Multi-Robot Navigation.
%         In: \emph{Adaptive Mobile Robotics: Proceedings of the 15th International
%         Conference on Climbing and Walking Robots and the Support Technologies
%         for Mobile Machines},
%         Baltimore, USA, 23 – 26 July 2012.
%         \doi{ 10.1142/9789814415958_0098}
% \end{bibenum}

% \section{Conference Posters}

% \begin{bibenum}

%     \item Terrones, A., Acuña, R., Certad-H, N., Fermín-León, L., and Fernández-López, G.
%         Local distributed control for multi-robot navigation. 15th International
%         Conference on Climbing and Walking Robots (CLAWAR 2012), Baltimore. July 26-26, 2012.
%         Poster abstract.

% \end{bibenum}

% \section{Student Advising}
%
% \begin{bibsection}
%     \item \textbf{Miguel Veloso}\\
%         Undergraduate student in Electronic Engineering, Simón Bolívar University.\\
%         Thesis topic: \textit{Development of a stabilization system for an inverted
%         pendulum, using a control system based on fluctuations inherent in human
%         motor control.} With honors. March 2015.
%
%     \item \textbf{Mauricio Marcano}\\
%         Undergraduate student in Electronic Engineering, Simón Bolívar University.\\
%         Thesis topic: \textit{Instrumentation and control of a superficial marine
%         vehicle used in bathymetric surveys.} With honors. March 2015.
%
%     \item \textbf{Daniel López}\\
%         Undergraduate student in Electronic Engineering, Simón Bolívar University.\\
%         Thesis topic: \textit{Development of monitoring nodes for a wireless
%         sensor network.} April, 2015.
%
%     \item \textbf{Alejandro Sánchez}\\
%         Undergraduate student in Electronic Engineering, Simón Bolívar University.\\
%         Thesis topic: \textit{Onmidirectional vision system for the mobile robot
%         AmigoBot.} September, 2015.
%
%     \item \textbf{Julio Colmenares}\\
%         Undergraduate student in Electronic Engineering, Simón Bolívar University.\\
%         Thesis topic: \textit{Development of a simulator for the RoboCup SSL
%         competition using ROS and V-REP.} With honors. September, 2015.
%
%      \item \textbf{Annybell Villarroel}\\
%         Undergraduate student in Electronic Engineering, Simón Bolívar University.\\
%         Thesis topic: \textit{Development of a 3D mapping system for inspection
%         of remote locations with a quadrotor.} With honors. September, 2015.
%
%      \item \textbf{Adrian Zuliani}\\
%         Undergraduate student in Electronic Engineering, Simón Bolívar University.\\
%         Thesis topic: \textit{Electronic design of a 4-wheeled omnidirectional
%         robot for the RoboCup SSL competition.} September, 2015.
%
%     \item \textbf{Alejandro R. Pérez M.}\\
%         Undergraduate student in Electronic Engineering, Simón Bolívar University.\\
%         Thesis topic: \textit{Development of a stability and altitude control
%         system for a quadrotor.} With honors. April, 2016.
%
%     \item \textbf{Jorge Pérez}\\
%         Undergraduate student in Electronic Engineering, Simón Bolívar
% University.\\
%         Thesis topic: \textit{Development of an FPGA-based embedded control
% system for an omnidirectional robot.} With honors. April, 2016.
%
%     \item \textbf{Gabriel Marzinotto.}\\
%         Undergraduate student in Electronic Engineering, Simón
% Bolívar University. Exchange Student at Telecom SudParis.\\
%         Thesis topic: \textit{Automatic analysis of handwritten text for aid
% and patiente monitoring.} In course.
% \end{bibsection}

% \section{Service}
% Open-source projects used for computer architecture courses:
% \begin{innerlist}
%     \item \href{https://bitbucket.org/NHT/os-demoqe128/wiki/Home}{OS-DEMOQE}: A
%     Operating System for the MC9S08QE128 microcontroller.
%     \item \href{https://bitbucket.org/NHT/os-twr/wiki/Home}{OS-TWR}: A Operating
%     System for the MCF51CN128 microcontroller.
%     \item \href{https://github.com/angelterrones/antares}{Antares}: Implementation
%     of the MIPS I RISC Processor.
%     \item \href{https://github.com/angelterrones/algol}{Algol}: Microarchitectural
%     Implementation of RV32IM ISA, using MyHDL python library.
%     \item \href{https://github.com/angelterrones/v-algol}{Algol}: Microarchitectural
%     Implementation of RV32I ISA using verilog.
% \end{innerlist}
%
% \halfblankline
%
% Advisor of the student robotic group \textit{FutBot USB}.
%
% \halfblankline
%
% Robotics Competition \textbf{CCSBOT 2013} (Universidad Simón Bolívar, 2013)
% \begin{innerlist}
%     \item Advisor of one team, representing the Universidad Simón Bolívar (USB).
% \end{innerlist}
%
% \halfblankline
%
% Robotics Competition \textbf{UNETBOTS 2015} (Universidad Nacional Experimental del Táchira, 2015)
% \begin{innerlist}
%     \item Advisor of two teams representing the Universidad Simón Bolívar (USB).
% \end{innerlist}
%
% \halfblankline

% \section{Projects}
% Personal projects:
% \begin{innerlist}
%     % \item \href{https://github.com/angelterrones/antares}{Antares}: Implementation
%     % of the MIPS I RISC Processor.
%     % \item \href{https://github.com/angelterrones/algol}{V-Algol}: Microarchitectural
%     % implementation of RV32IM ISA using verilog.
%     % \item \href{https://github.com/angelterrones/altair}{Altair}: Microarchitectural
%     % implementation of RV32IM ISA using nMigen.
%     % \item \href{https://github.com/angelterrones/bellatrix}{Bellatrix}: Microarchitectural
%     % implementation of RV32IM ISA using nMigen (pipelined).
%     \item \href{https://github.com/AngelTerrones/GA-render}{GA-render}: Image
%     approximation with polygons, using a genetic algorithm.
%     \item \href{https://github.com/AngelTerrones/GA-render-cuda}{GA-render-cuda}:
%     Image approximation with polygons, using a genetic algorithm (CUDA version).
% \end{innerlist}

Contractor projects:
\begin{innerlist}
\item Design an IP camera (2018 - 2019).
  \begin{innerlist}
  \item \textbf{Position}: Project engineer.
  \item \textbf{Description}: Develop the firmware for an IP camera (under NDA).
  \end{innerlist}
\item Adapt a code base to a new device (2016).
  \begin{innerlist}
  \item \textbf{Position}: Project engineer.
  \item \textbf{Description}: Port an existing code base to a new device developed by
    Intel. Project developed for MiOS Ltd (under NDA).
  \end{innerlist}
\item Design of a cleaning robot for photovoltaic panels (2015 - 2016).
  \begin{innerlist}
  \item \textbf{Position}: Project engineer.
  \item \textbf{Description}: Research and state of the art about cleaning robots for
    photovoltaic panels, and conceptual design of a solution. Project developed for ADVANCE SRN CORP.
  \end{innerlist}
\item Design a device for power line inspection (2013 - 2014).
  \begin{innerlist}
  \item \textbf{Position}: Project engineer.
  \item \textbf{Description}: Research and state of the art about robots for
    power line inspection, and elaboration of a conceptual design for a
    solution. Project developed for FUNDELEC.
  \end{innerlist}
\item Automatization of the P\&G's Baby Care Laboratory (2013).
  \begin{innerlist}
  \item \textbf{Position}: Project engineer.
  \item \textbf{Description}: Automatization of a set of mannequins and pumps
    (under NDA).
  \end{innerlist}
\end{innerlist}

\section{Software Skills}  % Hardware and
% Embedded and Real\-/time Systems:
% \begin{innerlist}
%     \item Software and hardware development with several MCU platforms (e.g.,
%     Freescale/NXP, Atmel, STM32)
%     \item FPGA development: Xilinx ISE, Xilinx Vivado, Verilator, Icarus Verilog, nMigen.
% \end{innerlist}
%
% \halfblankline

Computer Programming:
%
\begin{innerlist}
    \item C, C$++$, Python, UNIX shell scripting, GNU make, CMake.
\end{innerlist}

\halfblankline

Numerical Analysis:
%
\begin{innerlist}
    \item \Matlab.
    % \item Numpy.
\end{innerlist}

% \halfblankline

% Version Control and Software Configuration Management:
% %
% \begin{innerlist}
%     \item DVCS (Mercurial, Git)
% \end{innerlist}
%
% \halfblankline

% Operating Systems:
% %
% \begin{innerlist}
%     \item Microsoft Windows family, Linux (Ubuntu family, Arch Linux)
% \end{innerlist}

% \section{References Available to Contact}

% \textbf{Gerardo Fernández-López, Ph.D.}\\
% Caracas, Venezuela\\
% (e\-/mail:~\href{mailto:gfernandez@usb.ve}{gfernandez@usb.ve}; phone:~+58-412-903-4707)
% %
% \begin{innerlist}
%     \item Associate Professor,
%         Department of Electronic and Circuits, \href{http://www.usb.ve/}{Simón Bolívar University}

%     \item Head and Co-founder of \href{http://www.labc.usb.ve/mecatronica/}{R\&D Mechatronics Group}
% \end{innerlist}

% \halfblankline

% \textbf{Juan Carlos Grieco, Ph.D.}\\
% Caracas, Venezuela\\
% (e\-/mail:~\href{mailto:jcgrieco@usb.ve}{jcgrieco@usb.ve}; phone:~+58-414-462-4210)
% %
% \begin{innerlist}
%     \item Associate Professor,
%         Department of Electronic and Circuits, \href{http://www.usb.ve/}{Simón Bolívar University}

%     \item Head and Co-founder of \href{http://www.labc.usb.ve/mecatronica/}{R\&D Mechatronics Group}
% \end{innerlist}


%% The ``More Info'' section may not be necessary; make sure it's short
%% so it doesn't prevent people from seeing references available to
%% contact.
%\section{More Information}
%
%More information and auxiliary documents can be found at\\%
%\url{http://www.tedpavlic.com/facjobsearch/}.

\end{document}

%%%%%%%%%%%%%%%%%%%%%%%%%% End CV Document %%%%%%%%%%%%%%%%%%%%%%%%%%%%%

%----------------------------------------------------------------------%
% The following is copyright and licensing information for
% redistribution of this LaTeX source code; it also includes a liability
% statement. If this source code is not being redistributed to others,
% it may be omitted. It has no effect on the function of the above code.
%----------------------------------------------------------------------%
% Copyright (c) 2007, 2008, 2009, 2010, 2011 by Theodore P. Pavlic
%
% Unless otherwise expressly stated, this work is licensed under the
% Creative Commons Attribution-Noncommercial 3.0 United States License. To
% view a copy of this license, visit
% http://creativecommons.org/licenses/by-nc/3.0/us/ or send a letter to
% Creative Commons, 171 Second Street, Suite 300, San Francisco,
% California, 94105, USA.
%
% THE SOFTWARE IS PROVIDED "AS IS", WITHOUT WARRANTY OF ANY KIND, EXPRESS
% OR IMPLIED, INCLUDING BUT NOT LIMITED TO THE WARRANTIES OF
% MERCHANTABILITY, FITNESS FOR A PARTICULAR PURPOSE AND NONINFRINGEMENT.
% IN NO EVENT SHALL THE AUTHORS OR COPYRIGHT HOLDERS BE LIABLE FOR ANY
% CLAIM, DAMAGES OR OTHER LIABILITY, WHETHER IN AN ACTION OF CONTRACT,
% TORT OR OTHERWISE, ARISING FROM, OUT OF OR IN CONNECTION WITH THE
% SOFTWARE OR THE USE OR OTHER DEALINGS IN THE SOFTWARE.
%----------------------------------------------------------------------%
